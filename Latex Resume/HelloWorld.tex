\documentclass[UTF8]{ctexart}

\usepackage{graphicx}

\newtheorem{thm}{定理}

\title{杂谈勾股定理}
\author{张三}
\date{\today}

\bibliographystyle{plain}

\begin{document}

\maketitle

\begin{abstract}
这是一篇关于勾股定理的小短文
\end{abstract}

\tableofcontents
\section{勾股定理在古代}

创作也能到那么高端\footnote{欧几里得,约公元前330--275年。},\emph{被那么多人崇拜!}要是能重来,我要选李白,几百年前做的好坏没那么多人猜。至少我还能写些诗来崇拜,逗逗女孩。哈哈哈哈哈哈哈哈哈哈哈哈哈哈哈哈哈哈哈哈哈哈哈哈哈哈哈哈哈哈哈

\begin{thm}[勾股定理]
直角三角形斜边的平方等于两腰的平方和。
\end{thm}

\begin{equation}
a^2 + b^2 = c^2 * 90^\circ
\end{equation}

\begin{figure}[ht]
\centering
\includegraphics[height = 5cm]{hello.png}
\caption{这是一张经过高斯模糊的图片}
\label{fig:hello}
\end{figure}

\begin{table}[H]
\begin{tabular}{|rrr|}
\hline
直角边$a$ &直角边$b$ &斜边$c$\\
\hline
3&4&5\\
5&12&13\\
\hline
\end{tabular}
\end{table}

\begin{quote}
\zihao{-5}\kaishu 在那遥远的地方
\end{quote}

我的哈哈哈哈哈哈哈

\section{勾股定理的近代形式}
\bibliography{math}
\end{document}